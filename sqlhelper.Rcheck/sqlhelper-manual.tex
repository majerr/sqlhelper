\nonstopmode{}
\documentclass[a4paper]{book}
\usepackage[times,inconsolata,hyper]{Rd}
\usepackage{makeidx}
\usepackage[utf8]{inputenc} % @SET ENCODING@
% \usepackage{graphicx} % @USE GRAPHICX@
\makeindex{}
\begin{document}
\chapter*{}
\begin{center}
{\textbf{\huge Package `sqlhelper'}}
\par\bigskip{\large \today}
\end{center}
\inputencoding{utf8}
\ifthenelse{\boolean{Rd@use@hyper}}{\hypersetup{pdftitle = {sqlhelper: Easier SQL integration}}}{}\ifthenelse{\boolean{Rd@use@hyper}}{\hypersetup{pdfauthor = {Matthew Roberts}}}{}\begin{description}
\raggedright{}
\item[Title]\AsIs{Easier SQL integration}
\item[Version]\AsIs{0.1.2.9000}
\item[Description]\AsIs{Manage database connections and SQL execution.}
\item[Depends]\AsIs{R (>= 4.1.0)}
\item[Imports]\AsIs{DBI, yaml, rappdirs, stringr, glue, dplyr, pool, methods,
tibble, tidyr, purrr (>= 1.0.0), sf, spData, rlang}
\item[License]\AsIs{GPL (>= 3)}
\item[Encoding]\AsIs{UTF-8}
\item[RoxygenNote]\AsIs{7.2.1}
\item[Roxygen]\AsIs{list(markdown = TRUE)}
\item[Suggests]\AsIs{rmarkdown, knitr, testthat (>= 3.0.0), odbc, RSQLite,
RPostgres, RMariaDB, bigrquery}
\item[Config/testthat/edition]\AsIs{3}
\item[VignetteBuilder]\AsIs{knitr}
\item[URL]\AsIs{}\url{https://majr-red.github.io/sqlhelper/dev}\AsIs{,
}\url{https://github.com/majr-red/sqlhelper}\AsIs{}
\item[BugReports]\AsIs{}\url{https://github.com/majr-red/sqlhelper/issues}\AsIs{}
\item[NeedsCompilation]\AsIs{no}
\item[Author]\AsIs{Matthew Roberts [aut, cre]}
\item[Maintainer]\AsIs{Matthew Roberts }\email{matthew.roberts@levellingup.gov.uk}\AsIs{}
\end{description}
\Rdcontents{\R{} topics documented:}
\inputencoding{utf8}
\HeaderA{config\_examples}{Examples of yaml configurations for database connections}{config.Rul.examples}
%
\begin{Description}\relax
Provides example configurations for several databases and a range of options
\end{Description}
%
\begin{Usage}
\begin{verbatim}
config_examples(filename = NA)
\end{verbatim}
\end{Usage}
%
\begin{Arguments}
\begin{ldescription}
\item[\code{filename}] A string. If supplied, examples are written to a file with this name.
\end{ldescription}
\end{Arguments}
%
\begin{Details}\relax
Irrespective of whether a filename is supplied, yaml configuration examples
will be returned invisibly as a single string and printed if the session is
interactive.
\end{Details}
%
\begin{Value}
A yaml string of database configuration examples, invisibly.
\end{Value}
%
\begin{Examples}
\begin{ExampleCode}
config_examples()

# to write the examples to a file called 'examples.yml'
## Not run:  config_examples("examples.yml")

\end{ExampleCode}
\end{Examples}
\inputencoding{utf8}
\HeaderA{connect}{(Re-)Establish connections to databases}{connect}
%
\begin{Description}\relax
Closes any open connections, reads config files as directed by
\code{config\_filename} and \code{exclusive}, and creates new connections from the
descriptions in those files.
\end{Description}
%
\begin{Usage}
\begin{verbatim}
connect(config_filename = NA, exclusive = FALSE)
\end{verbatim}
\end{Usage}
%
\begin{Arguments}
\begin{ldescription}
\item[\code{config\_filename}] String. The full name and path of a configuration
file, or "site", or "user", or "example", or \code{NA} (the default). Cannot be
\code{NA} if \code{exclusive = TRUE}.

\item[\code{exclusive}] Logical. If \code{TRUE}, the file named by \code{config\_filename} is
treated as the only config file. Site and user level files are not read.
This parameter is ignored if \code{config\_filename} is missing.
\end{ldescription}
\end{Arguments}
%
\begin{Details}\relax
If \code{exclusive=FALSE} (the default), configuration files will be sought in the
directory returned by \code{\LinkA{rappdirs::site\_config\_dir()}{rappdirs::site.Rul.config.Rul.dir()}}, the directory returned
by \code{\LinkA{rappdirs::user\_config\_dir()}{rappdirs::user.Rul.config.Rul.dir()}}, and finally a file named by
\code{config\_filename} (if not \code{NA}). If elements of those files conflict, later
files overwrite the elements of earlier files.

if \code{exclusive=TRUE}, only 1 file will be read, as indicated by the
\code{config\_filename} parameter.
\begin{itemize}

\item{} if \code{config\_filename = "site"}, a config file called
\code{sqlhelper\_db\_conf.yml} will be sought in the directory returned by
\code{\LinkA{rappdirs::site\_config\_dir()}{rappdirs::site.Rul.config.Rul.dir()}}
\item{} if \code{config\_filename = "user"}, a config file called
\code{sqlhelper\_db\_conf.yml} will be sought in the directory returned by
\code{\LinkA{rappdirs::user\_config\_dir()}{rappdirs::user.Rul.config.Rul.dir()}}
\item{} if \code{config\_filename} is not \code{NULL} (but not "site" or "user"), it is
assumed to name a file.

\end{itemize}


\code{vignette("connections")} explains how to write a
config file and how to access the created connections.
\end{Details}
%
\begin{Value}
\code{NULL}, invisibly
\end{Value}
%
\begin{Examples}
\begin{ExampleCode}
## Not run: 
# Search for config files in rappdirs::site_config_dir(),
# rappdirs::user_config_dir(), and read `my_connections.yml` in the current
# working directory
connect("my_connections.yml")

# Read only `my_connections.yml` in the current working directory
connect("my_connections.yml", exclusive=TRUE)

## End(Not run)
\end{ExampleCode}
\end{Examples}
\inputencoding{utf8}
\HeaderA{connection\_info}{Browse available connections}{connection.Rul.info}
%
\begin{Description}\relax
Provides information about created connections.
\end{Description}
%
\begin{Usage}
\begin{verbatim}
connection_info(name_str = ".*", exact = TRUE)
\end{verbatim}
\end{Usage}
%
\begin{Arguments}
\begin{ldescription}
\item[\code{name\_str}] A regular expression to be used to identify connection names
to include. The default ('.*') returns all of them.

\item[\code{exact}] TRUE or FALSE. Should \code{name\_str} match the name of a connection
exactly? TRUE will identify only 1 connection if name\_str does not contain
any metacharacters
\end{ldescription}
\end{Arguments}
%
\begin{Value}
Null, or a tibble with 1 row per identified connection and the
following fields:

\begin{description}

\item[name] identifier (character)
\item[description] a description of the connection, if found in the conf file (character)
\item[live] is this connection valid and live? (logical)
\item[driver] the name of the driver function (character)
\item[conn\_str] the string used to parameterize the connection (character)
\item[pool] is this a pool connection? (logical)

\end{description}


If no connection names matched \code{name\_str}, the tibble will be empty. If
no connections have been configured, \code{NULL} is returned.
\end{Value}
\inputencoding{utf8}
\HeaderA{default\_conn}{Return the default connection}{default.Rul.conn}
%
\begin{Description}\relax
A convenience wrapper around \code{live\_connection()} and \code{get\_default\_conn\_name()}
\end{Description}
%
\begin{Usage}
\begin{verbatim}
default_conn()
\end{verbatim}
\end{Usage}
%
\begin{Value}
A database connection as returned by \code{DBI::dbConnect()}
\end{Value}
\inputencoding{utf8}
\HeaderA{disconnect}{Close all connections and remove them from the connections cache}{disconnect}
%
\begin{Description}\relax
Close all connections and remove them from the connections cache
\end{Description}
%
\begin{Usage}
\begin{verbatim}
disconnect()
\end{verbatim}
\end{Usage}
\inputencoding{utf8}
\HeaderA{is\_connected}{Test whether a database is connected}{is.Rul.connected}
\aliasA{not\_connected}{is\_connected}{not.Rul.connected}
%
\begin{Description}\relax
Test whether a database is connected
\end{Description}
%
\begin{Usage}
\begin{verbatim}
is_connected(conn_name)

not_connected(conn_name)
\end{verbatim}
\end{Usage}
%
\begin{Arguments}
\begin{ldescription}
\item[\code{conn\_name}] Character. The name of a connection (run
\code{\LinkA{connection\_info()}{connection.Rul.info}} for options)
\end{ldescription}
\end{Arguments}
%
\begin{Details}\relax
\code{not\_connected} is provided for readability, and is identical to \code{!is\_connected()}.
\end{Details}
%
\begin{Value}
Logical, or NULL if \code{conn\_name} does not identify exactly 1
connection
\end{Value}
\inputencoding{utf8}
\HeaderA{live\_connection}{Return the named connection or NULL}{live.Rul.connection}
%
\begin{Description}\relax
Return the named connection or NULL
\end{Description}
%
\begin{Usage}
\begin{verbatim}
live_connection(conn_name)
\end{verbatim}
\end{Usage}
%
\begin{Arguments}
\begin{ldescription}
\item[\code{conn\_name}] Chr. The name of the live connection you want (use
\LinkA{connection\_info}{connection.Rul.info} to get names of available connections).
\end{ldescription}
\end{Arguments}
%
\begin{Value}
A live connection to a database, or NULL (invisibly) if
\code{conn\_name} is not the name of a live connection
\end{Value}
\inputencoding{utf8}
\HeaderA{prepare\_sql}{prepare queries and meta data for execution}{prepare.Rul.sql}
%
\begin{Description}\relax
Except for \code{sql}, parameters are default values to be used when none are
supplied in \code{sql} (e.g. when \code{sql} is a tibble returned by \code{\LinkA{read\_sql()}{read.Rul.sql}}).
\end{Description}
%
\begin{Usage}
\begin{verbatim}
prepare_sql(
  sql,
  quotesql = "yes",
  values = parent.frame(),
  execmethod = "get",
  geometry = NA,
  default.conn = default_conn()
)
\end{verbatim}
\end{Usage}
%
\begin{Arguments}
\begin{ldescription}
\item[\code{sql}] An optionally-named list or character vector containing sql
commands, or a tibble returned by \code{\LinkA{read\_sql()}{read.Rul.sql}}

\item[\code{quotesql}] "yes" or "no" - should interpolated characters be quoted by
default? Anything that isn't "no" is treated as "yes".

\item[\code{values}] An environment containing variables to interpolate into the
SQL. Pass any object that is not an environment (e.g. "no", NA, FALSE or
NULL) if interpolation is to be skipped, or another environment containing
values to interpolate to avoid using \code{.GlobalEnv}.

\item[\code{execmethod}] One of "get", "execute", "sendq", "sends" or "spatial" -
which method should be used to execute the query? "get" means
\code{\LinkA{DBI::dbGetQuery()}{DBI::dbGetQuery()}}; "execute" means \code{\LinkA{DBI::dbExecute()}{DBI::dbExecute()}}; "sendq" means
\LinkA{DBI::dbSendQuery}{DBI::dbSendQuery}; "sends" means \code{\LinkA{DBI::dbSendStatement()}{DBI::dbSendStatement()}}; "spatial"
means \code{\LinkA{sf::st\_read()}{sf::st.Rul.read()}}.

\item[\code{geometry}] If \code{execmethod} is "spatial", which column contains the
geometry? Ignored if \code{execmethod} is not "spatial".

\item[\code{default.conn}] Either the name of a sqlhelper connection, or a database
connection returned by \code{\LinkA{DBI::dbConnect()}{DBI::dbConnect()}}, or NA. This connection is only
used by \code{\LinkA{glue::glue\_sql()}{glue::glue.Rul.sql()}} to interpolate SQL strings, it is not used to
execute any SQL code.
\end{ldescription}
\end{Arguments}
%
\begin{Details}\relax
The \code{default\_conn} parameter may be used to supply a connection object that
is not a configured sqlhelper connection which can then be used to interpolate
quoted strings.
\end{Details}
%
\begin{Value}
A tibble containing 1 row per query with the following fields:
\begin{description}

\item[qname] character. A name for this query
\item[quotesql] "yes" or "no". Should parameterized character values be quoted for this query?
\item[interpolate] "yes" or "no". Should this query be parameterized with values from R?
\item[execmethod] The method to execute this query.
One of "get" (\code{\LinkA{DBI::dbGetQuery()}{DBI::dbGetQuery()}}), "execute" (\code{\LinkA{DBI::dbExecute()}{DBI::dbExecute()}}), "sendq" (\code{\LinkA{DBI::dbSendQuery()}{DBI::dbSendQuery()}}), "sends" (\code{\LinkA{DBI::dbSendStatement()}{DBI::dbSendStatement()}}) or "spatial" (\code{\LinkA{sf::st\_read()}{sf::st.Rul.read()}})
\item[geometry] character. If \code{execmethod} is "spatial", which is the geometry column?
\item[conn\_name] character. The name of the database connection to use for this query.
Must be the name of a configured sqlhelper connection.
\item[sql] The sql query as entered
\item[filename] The value of \code{file\_name}
\item[prepared\_sql] The sql query to be executed, i.e. with interpolations
and quoting in place

\end{description}

\end{Value}
\inputencoding{utf8}
\HeaderA{read\_sql}{Read a sql file and return it's contents as a tibble}{read.Rul.sql}
%
\begin{Description}\relax
Read a sql file and return it's contents as a tibble
\end{Description}
%
\begin{Usage}
\begin{verbatim}
read_sql(file_name, cascade = TRUE)
\end{verbatim}
\end{Usage}
%
\begin{Arguments}
\begin{ldescription}
\item[\code{file\_name}] Full name and path of a file to read

\item[\code{cascade}] If TRUE, fill the values of absent execution parameters with
the most recent present value. This enables you to set the connection name
once, for the first query in a file and use the same connection for all the
subsequent queries, for example.
\end{ldescription}
\end{Arguments}
%
\begin{Details}\relax
Multiple SQL queries in the file should be terminated by semi-colons (\AsIs{;}).

The values of \code{qname}, \code{quotesql}, \code{interpolate}, \code{execmethod}, \code{geometry},
and \code{conn\_name} in the output may be controlled with comments
immediately preceding each query:

\begin{alltt}-- qname = create_setosa_table
-- execmethod = execute
-- conn_name = sqlite_simple
CREATE TABLE iris_setosa as SELECT * FROM IRIS WHERE SPECIES = 'setosa';

-- qname = get_setosa_table
-- execmethod = get
-- conn_name = sqlite_simple
SELECT * FROM iris_setosa;
\end{alltt}


With the exception of \code{qname}, the value of each interpreted comment is
cascaded to subsequent queries. This means you may set values once for the
first query in the file and they will apply to all the queries thereafter.
\end{Details}
%
\begin{Value}
A tibble containing 1 row per query with the following fields:
\begin{description}

\item[qname] character. A name for this query
\item[quotesql] "yes" or "no". Should parameterized character values be quoted for this query?
\item[interpolate] "yes" or "no". Should this query be parameterized with values from R?
\item[execmethod] The method to execute this query.
One of "get" (\code{\LinkA{DBI::dbGetQuery()}{DBI::dbGetQuery()}}), "execute" (\code{\LinkA{DBI::dbExecute()}{DBI::dbExecute()}}), "sendq" (\code{\LinkA{DBI::dbSendQuery()}{DBI::dbSendQuery()}}), "sends" (\code{\LinkA{DBI::dbSendStatement()}{DBI::dbSendStatement()}}) or "spatial" (\code{\LinkA{sf::st\_read()}{sf::st.Rul.read()}})
\item[geometry] character. If \code{execmethod} is "spatial", which is the geometry column?
\item[conn\_name] character. The name of the database connection to use for this query.
Must be the name of a configured sqlhelper connection.
\item[sql] The sql query to be executed
\item[filename] The value of \code{file\_name}

\end{description}


See \code{\LinkA{prepare\_sql()}{prepare.Rul.sql}} for the treatment of missing values in the output and their
defaults.
\end{Value}
%
\begin{Examples}
\begin{ExampleCode}

library(sqlhelper)

fn <- system.file( "examples/read_sql.sql", package="sqlhelper" )
readLines( fn ) |> writeLines()

connect( system.file("examples/sqlhelper_db_conf.yml", package="sqlhelper") )
read_sql(fn)

\end{ExampleCode}
\end{Examples}
\inputencoding{utf8}
\HeaderA{runfiles}{Read, prepare and execute .SQL files}{runfiles}
\aliasA{run\_files}{runfiles}{run.Rul.files}
\keyword{SQL runners}{runfiles}
%
\begin{Description}\relax
Accepts a character vector of SQL file names and attempts to execute each one.
\end{Description}
%
\begin{Usage}
\begin{verbatim}
runfiles(filenames, cascade = TRUE, ..., include_params = FALSE)

run_files(filenames, cascade = TRUE, ..., include_params = FALSE)
\end{verbatim}
\end{Usage}
%
\begin{Arguments}
\begin{ldescription}
\item[\code{filenames}] name, or vector of names, of file(s) to be executed

\item[\code{cascade}] If TRUE, fill the values of absent execution parameters with
the most recent present value. This enables you to set the connection name
once, for the first query in a file and use the same connection for all the
subsequent queries, for example.

\item[\code{...}] Arguments to be passed to \code{\LinkA{runqueries()}{runqueries}}

\item[\code{include\_params}] \code{TRUE} or \code{FALSE}. Should the parameters be
included in the output?
\end{ldescription}
\end{Arguments}
%
\begin{Details}\relax
\code{\LinkA{runfiles()}{runfiles}} enables you to control the arguments accepted by \code{\LinkA{runqueries()}{runqueries}}
on a per-query basis, using interpreted comments in your sql file:

\begin{alltt}-- qname = create_setosa_table
-- execmethod = execute
-- conn_name = sqlite_simple
CREATE TABLE iris_setosa as SELECT * FROM IRIS WHERE SPECIES = 'setosa';

-- qname = get_setosa_table
-- execmethod = get
-- conn_name = sqlite_simple
SELECT * FROM iris_setosa;
\end{alltt}


Interpreted comments precede the sql query to which they refer. Interpretable
comments are:

\begin{description}

\item[qname] A name for this query
\item[quotesql] "yes" or "no" - should interpolated characters be quoted?
\item[interpolate] "yes" or "no" - should sql be interpolated?
\item[execmethod] One of "get", "execute", "sendq", "sends" or "spatial" -
which method should be used to execute the query? "get" means
\code{\LinkA{DBI::dbGetQuery()}{DBI::dbGetQuery()}}; "execute" means \code{\LinkA{DBI::dbExecute()}{DBI::dbExecute()}}; "sendq" means
\code{DBI::dbSendQuery}; "sends" means \code{\LinkA{DBI::dbSendStatement()}{DBI::dbSendStatement()}}; "spatial"
means \code{\LinkA{sf::st\_read()}{sf::st.Rul.read()}}.
\item[geometry] The name of a spatial column. Ignored if \code{execmethod} is not 'spatial'
\item[conn\_name] The name of a connection to execute this query against

\end{description}


All interpreted comments except \code{qname} are recycled within their file, meaning
that if you want to use the same values throughout, you need only set them for
the first query.

\begin{alltt}readLines(
  system.file("examples/cascade.sql",
                package="sqlhelper")
) |> writeLines()
#> Warning in file(con, "r"): file("") only supports open = "w+" and open = "w+b":
#> using the former
\end{alltt}

\end{Details}
%
\begin{Value}
A list of results of sql queries found in files
\end{Value}
%
\begin{SeeAlso}\relax
\code{\LinkA{read\_sql()}{read.Rul.sql}}, \code{\LinkA{prepare\_sql()}{prepare.Rul.sql}}

Other SQL runners: 
\code{\LinkA{runqueries}{runqueries}()}
\end{SeeAlso}
\inputencoding{utf8}
\HeaderA{runqueries}{Execute a sequence of SQL queries}{runqueries}
\aliasA{run\_queries}{runqueries}{run.Rul.queries}
\keyword{SQL runners}{runqueries}
%
\begin{Description}\relax
Accepts a character vector of SQL queries and runs each one
\end{Description}
%
\begin{Usage}
\begin{verbatim}
runqueries(sql, ..., default.conn = default_conn(), include_params = FALSE)

run_queries(sql, ..., default.conn = default_conn(), include_params = FALSE)
\end{verbatim}
\end{Usage}
%
\begin{Arguments}
\begin{ldescription}
\item[\code{sql}] An optionally-named list or character vector containing sql
strings, or a tibble returned by \code{\LinkA{read\_sql()}{read.Rul.sql}} or \code{\LinkA{prepare\_sql()}{prepare.Rul.sql}}

\item[\code{...}] Arguments to be passed to \code{prepare\_sql()}

\item[\code{default.conn}] Either the name of a sqlhelper connection, or a database
connection returned by \code{\LinkA{DBI::dbConnect()}{DBI::dbConnect()}}, or NA. This connection is only
used by \code{\LinkA{glue::glue\_sql()}{glue::glue.Rul.sql()}} to interpolate SQL strings, it is not used to
execute any SQL code.

\item[\code{include\_params}] \code{TRUE} or \code{FALSE}. Should the parameters be
included in the output?
\end{ldescription}
\end{Arguments}
%
\begin{Value}
\begin{itemize}

\item{} If \code{include\_params} is \code{FALSE} and the \code{sql} argument is a
vector, a list containing the results of each query; element names will be
taken from the \code{sql} argument.
\item{} If the length of the \code{sql} argument is 1 and is not named, the result of
that query is returned as-is (e.g. a data.frame), not as a 1-element list.
\item{} If \code{include\_params} is \code{TRUE}, a tibble is returned containing 1
row per query with the following fields:

\end{itemize}


\begin{description}

\item[qname] character. A name for this query
\item[quotesql] "yes" or "no". Should parameterized character values be quoted for this query?
\item[interpolate] "yes" or "no". Should this query be parameterized with values from R?
\item[execmethod] The method to execute this query.
One of "get" (\code{\LinkA{DBI::dbGetQuery()}{DBI::dbGetQuery()}}), "execute" (\code{\LinkA{DBI::dbExecute()}{DBI::dbExecute()}}), "sendq" (\code{\LinkA{DBI::dbSendQuery()}{DBI::dbSendQuery()}}), "sends" (\code{\LinkA{DBI::dbSendStatement()}{DBI::dbSendStatement()}}) or "spatial" (\code{\LinkA{sf::st\_read()}{sf::st.Rul.read()}})
\item[geometry] character. If \code{execmethod} is "spatial", which is the geometry column?
\item[conn\_name] character. The name of the database connection to use for this query.
Must be the name of a configured sqlhelper connection.
\item[sql] The sql query to be executed
\item[filename] The value of \code{file\_name}
\item[prepared\_sql] The sql query to be executed, i.e. with interpolations
and quoting in place
\item[result] The result of the query

\end{description}

\end{Value}
%
\begin{SeeAlso}\relax
\code{\LinkA{runfiles}{runfiles}}

Other SQL runners: 
\code{\LinkA{runfiles}{runfiles}()}
\end{SeeAlso}
%
\begin{Examples}
\begin{ExampleCode}
{
library(sqlhelper)

connect(
    system.file("examples/sqlhelper_db_conf.yml", package="sqlhelper"),
    exclusive=TRUE)

DBI::dbWriteTable( default_conn(),
                  "iris",
                  iris)

n <- 5

runqueries(
    c(top_n = "select * from iris limit {n}", # interpolation is controlled
                                              # with the 'values' argument
      uniqs = "select distinct species as species from iris")
)

## Use the execmethod parameter if you don't want to return results
runqueries("create table iris_setosa as select * from iris where species = 'setosa'",
          execmethod = 'execute')

runqueries("select distinct species as species from iris_setosa")
}
\end{ExampleCode}
\end{Examples}
\inputencoding{utf8}
\HeaderA{set\_default\_conn\_name}{Set/get the name of the default connection to use}{set.Rul.default.Rul.conn.Rul.name}
\aliasA{get\_default\_conn\_name}{set\_default\_conn\_name}{get.Rul.default.Rul.conn.Rul.name}
%
\begin{Description}\relax
Set/get the name of the default connection to use
\end{Description}
%
\begin{Usage}
\begin{verbatim}
set_default_conn_name(conn_name)

get_default_conn_name()
\end{verbatim}
\end{Usage}
%
\begin{Arguments}
\begin{ldescription}
\item[\code{conn\_name}] Character string. The name a connection
\end{ldescription}
\end{Arguments}
%
\begin{Value}
\code{get} returns the name of the default connection; \code{set}
returns \code{NULL}, invisibly.
\end{Value}
\printindex{}
\end{document}
